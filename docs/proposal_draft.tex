\documentclass{article}

\usepackage{hyperref}

\author{Ethan Hawk}
\title{FUSTAL - \textbf{FU}thark \textbf{STA}tistical \textbf{L}ibrary}

\begin{document}
\maketitle
\tableofcontents

\newpage
\section{Description}

FUSTAL aims to be a general purpose library for statistical computation.
This will include subroutines for many different types of statistical test,
and hopefully will include code to run different types of machine learning
models.

FUSTAL will be written in \href{https://futhark-lang.org/}{futhark}, a
functional programming language, which compiles down to CUDA and OpenCL.

FUSTAL as a library will also be able to be used from a variety of different
programming environments, including C and Python, due to the nature of the
futhark compiler.

\section{Why}

This project was chosen for a number of reasons. The first of which is that
futhark, the implementation language, is impressive in many regards. Not the
least of which is its performance. Futhark is able to compile to code that runs
on the GPU, which can often give code a very significant speed increase.

An additional benefit to using futhark for this task, as opposed to C, is that 
since futhark is a high-level functional programming language, there is little
to no reliance on the underling machine architecture. There is no need to worry
about architecture specific performance increases, as those are often just
incorporated into the futhark compiler itself.

\section{Background}

Based off of the publications section of the futhark website,
\href{https://futhark-lang.org/publications.html}{here}, there doesn't
appear to be any work currently done in the realm of general statistics in
this language at the time of writing. However, there is a paper that details
the use of the language to help analyze time series data, so there is some
attention in this subject area.

\section{Goals}

The primary goal for this project is to produce a statistics library that
could be used in real world scenarios. Being able to run multiple different
statistical tests on a dataset, along with more basic summary statistics.

One of the big ``reach'' goals would be to have this library be capable of
running machine learning algorithms. Implementing the core subroutines for
a neural network is a goal of the project.

\section{Customers/Mentors}

Deciding who is the customer for this particular project is somewhat challenging.
Since the code will be open source, theoretically anyone can be a customer.
I would say that the ideal customer is someone who wants a library that they
can use to speed up their internal statistical processes, and has the requisite
technological background to be able to integrate this library into said process.

As for mentors, I plan to have some friends, both in and out of college help
with the development of the project. Obviously not as the main development lead,
but more so as an extra set of eyes to make sure that the project is going into
a fruitful direction.

\section{Data Needed}

For this project, the data that is needed is testing data. This is to ensure
that the library is able to reproduce results seen in other statistics
libraries and software, such as R. Some example testing datasets that I am
considering are the ``built-in'' datasets in R. The primary reason for their
consideration is ease of access, as well as being the closest datasets that
could even be called a ``known'' quantity.

\section{Data Science/Statistical Techniques}

Various Data Science and Statistical Techniques will need to be understood,
and implemented in this project.

Including, but not limited to the following list:

\begin{itemize}
\item t-test
\item chi-square test
\item Mann-Whitney U test
\item Wilcoxon rank sum test
\item Pearson correlation coefficient
\item ANOVA test
\item Linear Regression
\end{itemize}

This is of course in conjunction with methods to calculate all other manner
of statistical values from the dataset as well, since those procedures
need to be implemented anyways to be able to run any of the tests.

\section{Timeline}

Assuming that I do not start work this semester, which is unlikely, the rough time table would be as follows:

\begin{center}
\begin{tabular}{p{.2\linewidth}|p{.6\linewidth}}
Week Number & Goal\\
\hline
3 & Have basic statistical procedures done and a testing framework ready for use\\
7 & Sanity check statistical procedures and aim to have all tests done. Prepare for work on Neural Networks\\
9 & Aim to have neural network code done and test cases written.\\
14 & Finish Poster \& Paper\\
\end{tabular}
\end{center}

\section{How to share Results?}

Once the project is completed, and the library is statistically sound, results
will be announced to the developers of futhark, so that there can be an inclusion
on their website. This will likely be through their IRC channel, where I will
take questions and further suggestions on how to improve the library. Hopefully
by this point there will be some benchmarks comparing FUSTAL to other similar
libraries and software suites.

\section{Ethical Issues?}

The overarching ethical issue for this project is the accuracy of results.
Care needs to be taken to ensure that the results are correct and reproducible.
This is why it is important to use datasets in testing that are well understood,
so that any errors that arise can be dealt with earlier, and avoids the
reporting of wrong results.

\end{document}